\documentclass{beamer}
\usepackage[utf8]{inputenc}
\usepackage{lmodern}
\usepackage[brazil]{babel}
\usetheme{Madrid}

% -------------------------------------------------
% Identidade Visual - Centro Paula Souza
% -------------------------------------------------
\definecolor{cpsred}{RGB}{153,0,0}      % Vermelho institucional
\definecolor{cpsgray}{RGB}{85,85,85}    % Cinza técnico

\usecolortheme[named=cpsred]{structure}
\setbeamercolor{title}{fg=white,bg=cpsred}
\setbeamercolor{frametitle}{fg=white,bg=cpsred}
\setbeamercolor{structure}{fg=cpsred}
\setbeamercolor{normal text}{fg=black,bg=white}
\setbeamercolor{itemize item}{fg=cpsred}

\usepackage{ragged2e}
\usepackage{hyperref}

\setbeamertemplate{headline}{} 

% -------------------------------------------------
% Rodapé padrão
% -------------------------------------------------
\setbeamertemplate{footline}[frame number]
\addtobeamertemplate{footline}{
  \hfill\usebeamercolor[fg]{frametitle}{\hspace{-1.5cm}\scriptsize Amanda Oliveira, Arthur Fudali, Diego Souza, Giovana Albanês, Igor Leite}\hspace{1cm}
}{}

% -------------------------------------------------
% Informações principais
% -------------------------------------------------
\title[Teste de Desempenho Contínuo, Orientado por Visão Computacional]{\textbf{Teste de Desempenho Contínuo, Orientado por Visão Computacional}}
\author{Amanda Oliveira, Arthur Fudali, Diego Souza, Giovana Albanês, Igor Leite}
\date{}

% -------------------------------------------------
% Documento
% -------------------------------------------------
\begin{document}

% Capa
\begin{frame}
    \centering
    \vspace{1cm}
    {\color{cpsred}\Huge\textbf{Teste de Desempenho Contínuo, Orientado por Visão Computacional}}\\[0.8cm]
    {\Large Amanda Oliveira, Arthur Fudali, Diego Souza, Giovana Albanês, Igor Leite}\\[0.4cm]
    \textcolor{cpsgray}{Centro Paula Souza}\\[0.2cm]
    \textcolor{cpsgray}{FATEC Registro}
\end{frame}

% Sumário
\begin{frame}{Sumário}
\tableofcontents
\end{frame}

% -------------------------------------------------
% SEÇÕES E SLIDES
% -------------------------------------------------

\section{Pitch}

\section{Problematização}
\begin{frame}{Problematização}
\justifying
\begin{enumerate}
  \item Diagnóstico tradicional de TDA é subjetivo e inacessível.
  \item Falta de ferramentas tecnológicas acessíveis.
  \item Dificuldade em mensurar o comportamento atencional em tempo real.
  \item Baixo engajamento em processos avaliativos tradicionais.
\end{enumerate}
\end{frame}

\section{Estado da Arte}
\begin{frame}{Estado da Arte}
\justifying
    \centering
    \includegraphics[width=\textwidth]{figuras/nasc.png}
    
    \vspace{0.4cm}
    \textbf{Nascimento e Menezes (2020)}\\
    {\small
    A relação entre a prática regular de videogames e atenção sustentada
    }
\end{frame}

\begin{frame}{Estado da Arte}
\justifying

\begin{columns}[T,onlytextwidth]

    \begin{column}{0.48\textwidth}
        \centering
        \includegraphics[width=\textwidth]{figuras/elb2020.png}
        
        \vspace{0.4cm}
        \textbf{Elbaum et al. (2020)}\\
        {\small
        Attention-Deficit/Hyperactivity Disorder (ADHD): Integrating the MOXO-dCPT with an Eye Tracker Enhances Diagnostic Precision
        }
    \end{column}

    \begin{column}{0.04\textwidth}
        \rule{0.5pt}{0.75\textheight}
    \end{column}

    \begin{column}{0.48\textwidth}
        \centering
        \includegraphics[width=\textwidth]{figuras/wie2024.png}

        \vspace{0.4cm}
        \textbf{Wiebe et al. (2024)}\\
        {\small
        Virtual reality-assisted prediction of adult ADHD based on eye tracking, EEG, actigraphy and behavioral indices: a machine learning analysis of independent training and test samples
        }
    \end{column}

\end{columns}

\end{frame}

\section{Objetivo}
\begin{frame}{Objetivo}
\justifying
\begin{enumerate}
  \item Integrar eleentos do TDC em tarefas em formato de jogo.
  \item Coletar métricas de atenção e comportamento visual em tempo real.
  \item Comparar automaticamente os dados com parâmetros de referência.
  \item Gerar feedback imediato ao usuário.
  \item Oferecer uma solução prática e acessível para triagem de TDA.
\end{enumerate}
\end{frame}

\section{Metodologia}

% SLIDE 1
\begin{frame}{Metodologia — Metodologia do Jogo}
\justifying

\begin{itemize}
    \item Três fases com músicas progressivamente mais agitadas.
    \item Adaptação de métricas do TDC tradicional.
    \item Métricas coletadas:
    \begin{itemize}
        \item Acertos
        \item Erros por omissão
        \item Erros por comissão
        \item Tempo de resposta
    \end{itemize}
    \item Variabilidade temporal das respostas
    \item Dois modos de jogo: \textbf{Normal} e \textbf{Treinamento}
    % TODO: adicionar figura ilustrativa do jogo
\end{itemize}

\end{frame}

% SLIDE 2
\begin{frame}{Metodologia — Funcionamento das Fases}
\justifying

\begin{columns}[T,onlytextwidth]

    %---------------------------------------
    \begin{column}{0.48\textwidth}
        \centering
        \includegraphics[width=\textwidth]{figuras/primeira-fase.png}
        
        \vspace{0.4cm}

        \textbf{Fase 1 - Fixação visual}
        \begin{itemize}
            \item Olhar 5 estrelas por 5s cada.
            \item Computa acerto, omissão e comissão.
        \end{itemize}
    \end{column}

    %---------------------------------------
    \begin{column}{0.04\textwidth}
        \rule{0.5pt}{0.75\textheight}
    \end{column}

    %---------------------------------------
    \begin{column}{0.48\textwidth}
        \centering
        \includegraphics[width=\textwidth]{figuras/segunda-fase.png}

        \vspace{0.4cm}

        \textbf{Fase 2 - Atenção dividida}
        \begin{itemize}
            \item 2 rodadas de 15s com objetos transitando.
            \item Criança identifica planetas via botões IoT.
            \item Acertos e erros registrados.
        \end{itemize}
    \end{column}

\end{columns}

\end{frame}

\begin{frame}{Metodologia — Funcionamento das Fases}
\justifying

\begin{columns}[T,onlytextwidth]

    %---------------------------------------
    \begin{column}{0.48\textwidth}
        \centering
        \includegraphics[width=\textwidth]{figuras/terceira-fase.png}
        
        \vspace{0.4cm}

        \textbf{Fase 3 - Alternância de foco}
        \begin{itemize}
            \item Alternar olhar entre dois alvos que ligam/desligam.
            \item Avalia rapidez na troca de foco.
        \end{itemize}
    \end{column}

    %---------------------------------------
    \begin{column}{0.04\textwidth}
        \rule{0.5pt}{0.75\textheight}
    \end{column}

    %---------------------------------------
    \begin{column}{0.48\textwidth}
        \centering
        \includegraphics[width=\textwidth]{figuras/fim-geral.png}

        \vspace{0.4cm}

        \textbf{Resultados da IA}
        \begin{itemize}
            \item Gera pré-diagnóstico: acima/dentro/abaixo do esperado.
            \item Modo treinamento alimenta o módulo de IA.
        \end{itemize}
    \end{column}

\end{columns}

\end{frame}

\section{Apresentação prática}

\section{Resultados}
\begin{frame}{Resultados}
\justifying
\centering

\vspace{0.5cm}
{\Large\textbf{Resultados Parciais}}

\begin{itemize}
    \item Sistema de \textbf{eye tracking funcional}.
    \item \textbf{Front-end do jogo concluído} e operando as primeiras fases.
    \item Terceira fase \textbf{em desenvolvimento}.
\end{itemize}

\centering

\vspace{0.5cm}
{\Large\textbf{Próximos Passos}}

\begin{itemize}
    \item Refinar a \textbf{precisão do eye tracking funcional}.
    \item Finalizar a \textbf{implementação da Fase 3}.
    \item Iniciar \textbf{testes da primeira versão} do jogo.
    \item Melhoria contínua do sistema e criação da IA.
\end{itemize}

\end{frame}

\section{Conclusão}

\begin{frame}{Conclusão}
\justifying

\vspace{0.5cm}
{\Large\textbf{Desafios identificados}}

\begin{itemize}
    \item Integração do eye tracking com a mecânica do jogo.
    \item Testes em diferentes condições ambientais.
    \item Validação da eficiência do sistema.
\end{itemize}

\vspace{0.5cm}
{\Large\textbf{Potencial do projeto}}

\begin{itemize}
    \item Grande capacidade de evolução com avanços em visão computacional.
    \item Possibilidade de tornar a triagem de TDA mais acessível e intuitiva.
\end{itemize}

\end{frame}

\end{document}
