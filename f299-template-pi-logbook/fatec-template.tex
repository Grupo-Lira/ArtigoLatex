%%%% fatec-article.tex, 2024/03/10

%% Classe de documento
\documentclass[
landscape,
  a4paper,%% Tamanho de papel: a4paper, letterpaper (^), etc.
  12pt,%% Tamanho de fonte: 10pt (^), 11pt, 12pt, etc.
  english,%% Idioma secundário (penúltimo) (>)
  brazilian,%% Idioma primário (último) (>)
]{article}

%% Pacotes utilizados
\usepackage[]{fatec-article}
\usepackage{setspace}

%% Processamento de entradas (itens) do índice remissivo (makeindex)
%\makeindex%

%% Arquivo(s) de referências
%\addbibresource{fatec-article.bib}

%% Início do documento
\begin{document}

% Seções e subseções
%\section{Título de Seção Primária}%

%\subsection{Título de Seção Secundária}%

%\subsubsection{Título de Seção Terciária}%

%\paragraph{Título de seção quaternária}%

%\subparagraph{Título de seção quinária}%

%\section*{Diário de Bordo}%
\section*{Instruções para o preenchimento}
\doublespacing
\begin{enumerate}
    \item O Diário de Bordo é usado para registrar atividades, progressos, ideias e desafios enfrentados em um projeto ou durante a rotina de trabalho. Serve como um registro cronológico e detalhado das operações diárias, facilitando a organização e o acompanhamento das tarefas.
    \doublespacing
    \item Durante o registro das atividades deve-se incluir detalhes como datas, horários, descrições de tarefas, nomes de participantes e observações relevantes.  Esta documentação contínua ajuda na avaliação do progresso de projetos ou atividades, permitindo ajustes e melhorias contínuas nos processos.
    \doublespacing
    \item Para evidenciar a realização das tarefas, você poderá utilizar a criação de anexos para adicionar anotações, fotos, prints, questionários, entre outros.
\end{enumerate}

\break

 \begin{table}[]
\centering
\begin{tabular}{|p{5cm}|l|l|l|p{8cm}|}
\hline
Nome da Atividade & Data de início & Data de término & Responsável pela atividade & Descrição da atividade realizada \\ \hline
Ideias de perguntas para pesquisa de campo        & 13/08/2025 & 13/08/2025 & Todos   & Brainstorming de perguntas para entrevistas com usuários. \\ \hline
Revisão das perguntas da pesquisa de campo        & 13/08/2025 & 13/08/2025 & Todos   & Revisão e refinamento das perguntas geradas no brainstorming. \\ \hline
Definição das tarefas de cada um                  & 19/08/2025 & 19/08/2025 & Todos   & Distribuição das tarefas entre os membros da equipe com base nas perguntas definidas. \\ \hline
Estudo sobre bibliotecas e sistemas de eye tracking existentes & 18/08/2025 & 26/08/2025 & Diego   & Pesquisa e análise de bibliotecas e sistemas de eye tracking disponíveis no mercado. \\ \hline
Estudo da biblioteca WebGazer.js & 26/08/2025 & 26/08/2025 & Diego   & Estudo detalhado da biblioteca WebGazer.js para possível utilização no projeto. \\ \hline
Desenvolvimento de testes com a biblioteca WebGazer.js & 26/08/2025 & 05/09/2025 & Diego   & Desenvolvimento de testes práticos utilizando a biblioteca WebGazer.js para avaliar sua funcionalidade. \\ \hline
Adição da pesquisa de campo no template           & 27/08/2025 & 27/08/2025 & Giovana & Inclusão das perguntas definidas no template de pesquisa de campo. \\ \hline
Revisão do diagrama de classes e de objetos       & 29/08/2025 & 29/08/2025 & Giovana & Revisão e atualização dos diagramas de classes e de objetos com base nas novas informações coletadas. \\ \hline
Conversão do artigo para LaTeX                    & 04/09/2025 & 04/09/2025 & Arthur  & Conversão do artigo para o formato LaTeX. \\ \hline
Submissão do artigo para correção                 & 04/09/2025 & 04/09/2025 & Arthur  & Submissão do artigo para o processo de correção. \\ \hline
Tratamento das coordenadas do jogador no backend  & 06/09/2025 & 14/09/2025 & Giovana & Tratamento das coordenadas do jogador no backend. \\ \hline
Reformulação das perguntas da pesquisa de campo   & 11/09/2025 & 11/09/2025 & Giovana & Reformulação das perguntas da pesquisa de campo com base no feedback recebido. \\ \hline
\end{tabular}
\end{table}

\break

 \begin{table}[]
\centering
\begin{tabular}{|p{5cm}|l|l|l|p{8cm}|}
\hline
Nome da Atividade & Data de início & Data de término & Responsável pela atividade & Descrição da atividade realizada \\ \hline
Criação do briefing                               & 15/09/2025 & 15/09/2025 & Amanda  & Criação do briefing para alinhamento das expectativas do projeto. \\ \hline
Vetorização da fase 2 do jogo                     & 16/09/2025 & 18/09/2025 & Igor  & Vetorização dos elementos gráficos da fase 2 do jogo. \\ \hline
Criação das personas                              & 18/09/2025 & 18/09/2025 & Amanda  & Criação de personas para guiar o desenvolvimento do projeto. \\ \hline
Aplicação do vetor na fase 2 do jogo                  & 20/09/2025 & 20/09/2025 & Igor  & Aplicação dos vetores criados na fase 2 do jogo. \\ \hline
Escrita e correção dos objetivos do artigo        & 06/10/2025 & 08/10/2025 & Giovana & Escrita e correção dos objetivos do artigo com base no feedback recebido. \\ \hline
Escrita e correção das metodologias do artigo     & 06/10/2025 & 29/10/2025 & Giovana & Escrita e correção das metodologias do artigo com base no feedback recebido. \\ \hline
Desenvolvimento do front-end da calibração        & 10/10/2025 & 10/10/2025 & Amanda  & Desenvolvimento do front-end da calibração. \\ \hline
Desenvolvimento do IoT para coleta de dados         & 10/10/2025 &  & Igor  & Desenvolvimento do IoT para coleta de dados do jogo. \\ \hline
Reunião de alinhamento                            & 16/10/2025  & 16/10/2025 & Todos   & Reunião para alinhamento das atividades e definição de próximos passos. \\ \hline
Reunião com o orientador do artigo                & 16/10/2025  & 16/10/2025 & Todos   & Reunião com o orientador do artigo para discutir o progresso e receber feedback. \\ \hline
Criação do sitemap & 17/10/2025 & 21/10/2025 & Diego   & Criação do sitemap para o projeto. \\ \hline
Desenvolvimento do front-end da fase 2 do jogo    & 18/10/2025  & 18/10/2025 & Amanda   & Desenvolvimento do front-end da fase 2 do jogo. \\ \hline
Criação do design da fase 3 do jogo               & 19/10/2025  & 20/10/2025 & Amanda   & Criação do design da fase 3 do jogo. \\ \hline
Correção da funcionalidade de configuração do volume & 19/10/2025  & 19/10/2025 & Amanda   & Correção da funcionalidade de configuração do volume no jogo. \\ \hline
Adição de tutorial da fase                         & 19/10/2025  & 19/10/2025 & Amanda   & Adição de tutorial das fases do jogo. \\ \hline

\end{tabular}
\end{table}

\break

 \begin{table}[]
\centering
\begin{tabular}{|p{5cm}|l|l|l|p{8cm}|}
\hline
Nome da Atividade & Data de início & Data de término & Responsável pela atividade & Descrição da atividade realizada \\ \hline
Implementação da biblioteca no projeto & 19/10/2025 & 19/10/2025 & Diego   & Implementação da biblioteca WebGazer.js no projeto. \\ \hline
Alinhamento sobre o projeto                        & 20/10/2025  & 20/10/2025 & Todos   & Alinhamento sobre o projeto e definição de próximos passos. \\ \hline
Documentação dos artefatos do projeto             & 20/10/2025  & 29/10/2025 & Giovana   & Documentação dos artefatos do projeto para registro e referência futura. \\ \hline
Correções do artigo                                & 20/10/2025  & 20/10/2025 & Arthur   & Realização das correções do artigo com base no feedback recebido. \\ \hline
Revisão do diagrama de redes                       & 20/10/2025  & 20/10/2025 & Arthur   & Revisão do diagrama de redes para garantir precisão e clareza. \\ \hline
Revisão do CANVAS                                 & 20/10/2025  & 20/10/2025 & Arthur   & Revisão do CANVAS para assegurar que todos os aspectos do projeto estão cobertos. \\ \hline
Desenvolvimento da fase 3 do jogo                 & 20/10/2025  & 20/10/2025 & Amanda   & Desenvolvimento da fase 3 do jogo. \\ \hline
Organização do design no figma                    & 20/10/2025  & 20/10/2025 & Amanda   & Organização para facilitar o acesso e a colaboração. \\ \hline
Integração back e front-end                       & 20/10/2025  & 26/10/2025 & Diego   & Integração do back-end e front-end do projeto para garantir funcionalidade completa. \\ \hline
Introdução do artigo                               & 20/10/2025 & 25/10/2025 & Arthur   & Escrita da introdução do artigo. \\ \hline
Edição do pitch do projeto                          & 20/10/2025 & 26/10/2025 & Igor   & Edição do pitch do projeto para apresentação. \\ \hline
Correções da analise Swot                         & 23/10/2025 & 23/10/2025 & Igor   & Realização das correções da análise SWOT com base no feedback recebido. \\ \hline
Criação do diagrama de casos de uso               & 23/10/2025 & 23/10/2025 & Arthur   & Criação do diagrama de casos de uso para o projeto. \\ \hline
Revisão da metodologia                             & 25/10/2025 & 25/10/2025 & Arthur   & Revisão da metodologia do artigo. \\ \hline
Ajuste da lógica da fase 1                         & 25/10/2025 & 25/10/2025 & Giovana   & Ajuste da lógica da fase 1 do jogo. \\ \hline

\end{tabular}
\end{table}

\break

 \begin{table}[]
\centering
\begin{tabular}{|p{5cm}|l|l|l|p{8cm}|}
\hline
Nome da Atividade & Data de início & Data de término & Responsável pela atividade & Descrição da atividade realizada \\ \hline
Pesquisa de bibliotecas para animação & 25/10/2025 & 25/10/2025 & Igor   & Início da pesquisa de bibliotecas para animação dos elementos gráficos do jogo. \\ \hline
Ajuste da lógica da fase 2                         & 26/10/2025 & 26/10/2025 & Giovana   & Ajuste da lógica da fase 2 do jogo. \\ \hline
Ajustes das telas finais e de feedback no figma    & 26/10/2025 & 26/10/2025 & Amanda   & Ajustes das telas finais e de feedback no Figma. \\ \hline
Inclusão de registros no template do Logbook & 26/10/2025 & 29/10/2025 & Amanda   & Início da inclusão de registros no template do Logbook para documentar o progresso do projeto. \\ \hline
Criação de rota para conexão da fase 2 com IoT & 28/10/2025 & 28/10/2025 & Giovana   & Criação de rota para conexão da fase 2 do jogo com o IoT para coleta de dados. \\ \hline
Ajustes no front-end da fase 2                        & 28/10/2025 & 29/10/2025 & Amanda   & Realização de ajustes no front-end da fase 2 do jogo para melhorar a experiência do usuário. \\ \hline
Revisão dos slides da apresentação final              & 28/10/2025 & 29/10/2025 & Amanda   & Revisão dos slides da apresentação final para garantir clareza e impacto. \\ \hline
\end{tabular}
\end{table}

\end{document}