%%%% fatec-article.tex, 2024/03/10

%% Classe de documento
\documentclass[
  a4paper,%% Tamanho de papel: a4paper, letterpaper (^), etc.
  12pt,%% Tamanho de fonte: 10pt (^), 11pt, 12pt, etc.
  english,%% Idioma secundário (penúltimo) (>)
  brazilian,%% Idioma primário (último) (>)
]{article}

%% Pacotes utilizados
\usepackage[]{fatec-article}
\usepackage{graphicx}
\usepackage{float}

\Author{1}{Name={Amanda de Oliveira Costa\\ Arthur Ribeiro Dias Fudali \\ Diego Baltazar de Souza Claudio \\ Giovana da Silva Albanês Santos \\ Igor Leite Gomes }}

\Author{2}{Name={\{ amanda.costa47@fatec.sp.gov.br  \}\\ \{ arthur.fudali@fatec.sp.gov.br \} \\ \{ diego.claudio@fatec.sp.gov.br\} \\ \{ giovana.santos30@fatec.sp.gov.br \} \\ \{ igor.gomes4@fatec.sp.gov.br \}}}

%% Definição das palavras-chaves/keywords
\Keyword{1}{TDA}{ADHD}
\Keyword{2}{atenção}{attention}
\Keyword{3}{rastreamento ocular}{eye tracking}
\Keyword{4}{gamificação}{gamification}
\Keyword{5}{ODS 3}{SDG 3}

%%%% Resumo no idioma primário (brazilian)
\begin{Abstract}[brazilian]%% Idioma (brazilian ou english)
O Transtorno de Déficit de Atenção (TDA) é uma condição neuropsicológica que pode
afetar significativamente o desempenho de crianças em diversas esferas da vida, como
estudos e relações interpessoais. O diagnóstico tradicional envolve avaliações
clínicas e testes padronizados, os quais nem sempre estão disponíveis de forma acessível ou
apresentam dados objetivos suficientes para um rastreio inicial eficaz. Este estudo propõe
uma abordagem tecnológica que integra uma plataforma digital gamificada, baseada no
Teste de Desempenho Contínuo (TDC), com técnicas de rastreamento ocular em tempo real.
A proposta tem como objetivo identificar possíveis indícios de TDA em crianças de 10 a 12 anos por meio da
análise de desempenho atencional durante a execução de tarefas gamificadas. O sistema
registra dados como acertos, tempo de reação, erros de omissão e comissão, variabilidade nas
respostas e padrões de fixação ocular, os quais são processados automaticamente. Com base
na comparação com bases de dados normativas, o sistema fornece um feedback
interpretativo ao usuário, podendo contribuir como ferramenta complementar de triagem
inicial. A acessibilidade da plataforma, que roda diretamente no navegador, e seu caráter
lúdico e autônomo tornam a solução promissora para ampliar o acesso ao rastreamento
precoce de indícios do transtorno. A proposta está alinhada ao terceiro Objetivo de
Desenvolvimento Sustentável (ODS) da Agenda 2030 da ONU, que visa garantir saúde e
bem-estar para todas as pessoas, promovendo inovações tecnológicas para uma saúde mais
inclusiva, eficiente e orientada por dados objetivos.
\end{Abstract}

%%%% Resumo no idioma secundário (english)
\begin{Abstract}[english]%% Idioma (brazilian ou english)
Attention Deficit Disorder (ADD) is a neuropsychological condition that can
significantly affect children's performance in various areas of life, such as
studies and interpersonal relationships. Traditional diagnosis involves clinical
assessments and standardized tests, which are not always readily available or
provide sufficient objective data for effective initial screening. This study proposes
a technological approach that integrates a gamified digital platform, based on the
Continuous Performance Test (CPT), with real-time eye-tracking techniques.
The aim is to identify possible signs of ADD in children aged 10 to 12 years through the
analysis of attentional performance during the execution of gamified tasks. The system
records data such as correct answers, reaction time, omission and commission errors, variability in
responses, and eye fixation patterns, which are processed automatically. Based
on comparison with normative databases, the system provides
interpretive feedback to the user, potentially contributing as a complementary screening tool
for initial assessment. The accessibility of the platform, which runs directly in the browser, and its
playful and autonomous nature make the solution promising for expanding access to
early screening for signs of the disorder. The proposal is aligned with the third Sustainable
Development Goal (SDG) of the UN's 2030 Agenda, which aims to ensure health and
well-being for all people, promoting technological innovations for a more
inclusive, efficient, and data-driven healthcare system.
\end{Abstract}

%% Processamento de entradas (itens) do índice remissivo (makeindex)
\makeindex%

%% Arquivo(s) de referências
\addbibresource{fatec-article.bib}

%% Início do documento
\begin{document}

% Seções e subseções
%\section{Título de Seção Primária}%

%\subsection{Título de Seção Secundária}%

%\subsubsection{Título de Seção Terciária}%

%\paragraph{Título de seção quaternária}%

%\subparagraph{Título de seção quinária}%

\section*{Introdução}%
\label{sect:intro}
A Organização das Nações Unidas (ONU) é uma instituição que visa estabelecer a paz,
segurança e desenvolvimento global. A ONU conta com 193 países membros que formam a
Assembleia Geral, responsável por desenvolver as políticas da organização. Em 2015, como
parte da Agenda 2030 para o Desenvolvimento Sustentável, foram criados 17 objetivos que
abrangem desde a melhoria da indústria até o aprimoramento da saúde da população. Esse
conjunto de metas constitui um plano de ação ambicioso para as pessoas, o planeta e a
prosperidade. Este trabalho visa contribuir para o terceiro objetivo, que busca garantir o
acesso à saúde e promover o bem estar, ao desenvolver uma plataforma digital gamificada
com o objetivo de auxiliar nos possíveis indícios do Transtorno de Déficit de Atenção (TDA)
em adultos.

O TDA é uma condição neurobiológica de causas genéticas, que afeta milhões de pessoas em todo o mundo. É caracterizada por sintomas de
desatenção, impulsividade, e, em alguns casos, hiperatividade, afetando significativamente o
desempenho acadêmico, profissional e social dos afetados. Embora o diagnóstico clínico do
TDA seja baseado tradicionalmente em entrevistas e questionários subjetivos, avanços
tecnológicos têm permitido o uso de ferramentas mais robustas e quantitativas para apoiar
esse processo. \textcite{BVS2014}
Entre essas ferramentas, destaca-se o \textit{eye tracking} (do português, rastramento ocular), esta é uma técnica
que consiste em usar o posicionamento dos olhos de uma pessoa para obter informações
sobre onde ela está olhando. Isso pode ser feito usando luzes infravermelhas, que calculam
exatamente onde a pessoa está olhando com base nas reflexões da luz na retina, ou por
meio de câmeras que monitoram visualmente a posição dos olhos e identificam sua direção.

Em ambientes controlados, o teste com \textit{eye tracking} envolve a realização de tarefas
padronizadas, nas quais o comportamento visual do participante é monitorado sem
interferência direta de um moderador. Essa abordagem objetiva permite uma coleta mais
confiável de dados, reduzindo o viés associado ao autorrelato e aumentando a credibilidade
da análise. Além disso, a comparação dos dados obtidos com padrões normativos permite
identificar desvios significativos no desempenho visual atencional, muitas vezes
imperceptíveis a métodos tradicionais.

A Inteligência Artificial (IA), é um campo da computação dedicado à criação de sistemas capazes de replicar a habilidade humana de executar tarefas que requerem percepção, raciocinio, aprendizado e tomada de decisão. A IA pode ser dividida em IA Forte e IA Fraca, \textcite{sep-chinese-room} define a IA Forte como programas que conseguem, além de entender a linguagem natural, também conseguem ter todas as características de raciocinio de um humano. A IA Fraca seria capaz de somente replicar tarefas específicas, e não possuir capacidade de raciocínio. A capacidade da IA de realizar tarefas é obtida através de algoritmos de aprendizagem profunda, processamento de linguagem natural e análise de dados.

O conceito de Internet das Coisas (do inglês, Internet of Things, ou IoT) vem da conexão de dispostivos fisícos (como sensores, dispositivos inteligêntes, eletrodomésticos, etc. ) à internet por meio de protocolos de comunicação, o que permite que os dados coletados sejam amplamente distribuídos e compartilhados de forma autônoma. A IoT integra computação em nuvem, comunicação de máquina e análise de dados para criar sistemas integrados e inteligentes. https://iot.ieee.org/
 
Quando aplicada em um teste diagnóstico, a técnica permite acompanhar com
precisão os movimentos dos olhos de um indivíduo durante a realização de tarefas
específicas, fornecendo dados objetivos e detalhados sobre onde, por quanto tempo e em
que sequência uma pessoa fixa seu olhar em determinados estímulos visuais. Estudos
demonstram que pessoas com TDA apresentam menor tempo de fixação e padrão visual
mais disperso ao realizarem testes de desempenho, sugerindo dificuldade em manter a
atenção sustentada e filtragem de estímulos irrelevantes \textcite{Lim2024}.

O Teste de Desempenho Contínuo (TDC) é uma medida padronizada amplamente
utilizada na neuropsicologia para avaliar métricas de atenção sustentada, impulsividade,
tempo de resposta e variabilidade dos tempos de reação. Trata-se de uma tarefa
computadorizada na qual o usuário responde e reage a estímulos apresentados
sequencialmente, permitindo medidas de desempenho atencional ao longo do tempo. Os
\textit{serious games}, ou jogos sérios, são aplicações digitais desenvolvidas com finalidades que
vão além do entretenimento, como educação, treinamento ou reabilitação cognitiva. No
contexto ne análises neuropsicológicas, eles vêm sendo utilizados como ferramentas
complementares aos testes tradicionais, oferecendo ambientes imersivos e interativos que
auxiliam no engajamento do usuário e a mensuração de habilidades cognitivas.

Com isso, a aplicação do \textit{eye tracking} em algo como os jogos sérios, podem servir não
apenas como ferramentas para complementar o diagnóstico clínico, mas também como um
potencial instrumento de triagem inicial. A análise dos dados com IA pode indicar o grau de
desatenção apresentado por um indivíduo em diferentes contextos e contribuir para decisões
clínicas mais embasadas. Assim, propomos por meio deste estudo a criação de um software
gamificado, que usa as informações obtidas pelo rastreamento ocular de possíveis afetados
pelo TDA e as processa, usando as métricas do TDC para gerar um resultado médio do
desempenho geral, fornecendo uma possível indicação para o transtorno.

\section*{OBJETIVO} \label{sect:obj}

Desenvolver uma plataforma digital gamificada, acessível por navegador, que utiliza
rastreamento ocular e métricas de desempenho atencional para auxiliar na triagem inicial de
indícios do Transtorno de Déficit de Atenção (TDA) em crianças de 10 a 12 anos.

\subsection*{OBJETIVOS ESPECÍFICOS} 

\begin{itemize}
    \item Integrar o Teste de Desempenho Contínuo (TDC) a um jogo digital com tarefas que
    simulam desafios atencionais.
    \item  Aplicar técnicas de rastreamento ocular em tempo real para registrar padrões de
    atenção durante a execução das tarefas.
    \item Processar automaticamente os dados coletados e compará-los com parâmetros
    normativos para gerar feedback ao usuário.
    \item Promover uma solução acessível, autônoma e alinhada ao ODS 3 da ONU, que amplie
    o acesso à triagem inicial de TDA.
\end{itemize}


\section*{ESTADO DA ARTE} \label{sect:estadoarte}

O diagnóstico do Transtorno de Déficit de Atenção (TDA) em adultos continua sendo
um desafio clínico significativo. Tradicionalmente, o processo diagnóstico baseia-se em
entrevistas e testes clínicos, autorrelato e relatos de informantes, instrumentos que, embora
úteis, podem ser afetados por viés retrospectivo, subjetividade do paciente, e simulação dos
testes, resultando em casos de falsos positivos ou negativos. Dessa forma, o interesse por
abordagens objetivas e tecnologicamente assistidas, que combinem dados neuropsicológicos
e comportamentais com técnicas de análises automatizadas, aumentou. Estudos recentes
têm investigado o uso de jogos digitais sérios como ferramentas de avaliação e treinamento
cognitivo, com foco na atenção contínua. Nascimento e Menezes (2020) exploraram a relação
entre a prática regular de videogames e o desempenho em atenção sustentada, avaliado
pelo Conners’ Continuous Performance Test II (CPT II), uma versão amplamente utilizada do
Teste de Desempenho Contínuo (TDC). Embora não tenham encontrado diferenças de
performance entre jogadores de videogames de ação, não ação e não jogadores, os autores
identificaram o sexo como uma variável relevante, pois notara diferença entre tempo de
reação e número de erros. O estudo destaca a complexidade das interações entre fatores
individuais e experiências digitais, sugerindo que a aplicação de jogos, mesmo quando
classificados como serious games, precisam de cuidados na metodologia e no controle de
variáveis, com o fim de evitar interferências no desempenho atencional.

Nesse contexto, Elbaum et al. (2020) exploraram o potencial diagnóstico da integração
entre o MOXO-dCPT (um teste de desempenho contínuo com fases estruturadas de distração
auditiva e visual) e dados de rastreamento ocular (eye tracking). O estudo contou com uma
amostra de 85 adultos (43 com diagnóstico formal de TDAH e 42 controles saudáveis) e
analisou o padrão de atenção visual ao longo de diferentes partes do teste. Os resultados
demonstraram que indivíduos com TDAH apresentaram maior tempo de fixação em áreas
irrelevantes da tela, particularmente em condições com distrações visuais, o que os autores
interpretaram como uma medida direta de distratibilidade atencional objetiva. Essa métrica
comportamental demonstrou maior poder discriminativo em comparação às melhores
pontuações tradicionais do MOXO. Além disso, os autores propuseram que as partes do teste
com distrações visuais poderiam ser utilizadas isoladamente, reduzindo o tempo do teste e
mantendo a precisão.

Avançando nesse campo, Wiebe et al. (2024) criaram uma solução diagnóstica que
envolve um ambiente de realidade virtual (VR), onde os participantes realizavam um teste de
desempenho imersivo sob a ocorrência de distrações simuladas em um ambiente 3D.
Durante a tarefa, foram coletados dados simultâneos de eye tracking, movimentos da
cabeça, eletroencefalograma (EEG) e desempenho atencional. O modelo de IA foi treinado
em um conjunto de 50 participantes e testado de forma independente em outro conjunto de
36 indivíduos. O modelo final, com apenas 11 variáveis selecionadas, alcançou 81\% de
acurácia, 78\% de sensibilidade e 83\% de especificidade no conjunto de teste.

Os estudos indicam que a utilização de tecnologias de rastreamento ocular, tarefas
cognitivas com análises embasadas dos dados representam um avanço significativo em
relação aos métodos tradicionais de diagnóstico. O uso de serious games para coleta de
dados de desempenho também é eficaz, como mostra o trabalho de Nascimento e Menezes.
O trabalho de Elbaum et al. oferece um modelo aplicável e eficiente ao integrar eye tracking
em um teste comercial já existente, o estudo de Wiebe et al. diferencia a proposta ao
incorporar realidade virtual, aprendizado de máquina e validação externa em amostras
independentes. Juntos, os estudos reforçam a ideia de que sistemas digitais automatizados
podem melhorar a precisão diagnóstica do TDAH em adultos.



\section*{METODOLOGIA} \label{sect:metodologia}

A \textit{landing page} do projeto foi desenvolvida em \textit{HTML}, responsável pela estrutura do conteúdo, \textit{CSS}, utilizado para o estilo visual, e \textit{JavaScript}, empregado na implementação da interatividade e do dinamismo da navegação. O sistema de rastreamento ocular foi implementado em \textit{JavaScript}, utilizando a biblioteca \textit{WebGazer.js} \cite{papoutsaki2016webgazer}, que contém um modelo capaz de se autocalibrar ao observar a interação dos visitantes com a página, treinando um mapeamento entre as características do olhar e as posições na tela. O tratamento das coordenadas oculares recebidas do \textit{frontend} foi realizado em \textit{JavaScript}, com o uso do \textit{Node.js} e do \textit{Socket.IO}, possibilitando a comunicação em tempo real com o \textit{frontend}, uma vez que depende das coordenadas enviadas por ele. Essa parte do \textit{backend} é responsável por analisar métricas como acertos, erros por omissão, erros por comissão, tempo de reação e variabilidade temporal das respostas, dados que serão utilizados para compor o feedback individual de cada usuário. Por fim, o jogo \textit{web} foi desenvolvido em \textit{TypeScript}, utilizando o \textit{framework} \textit{Next.js}, o que proporcionou um código mais robusto, organizado e uma experiência de uso moderna e fluida.

A metodologia fundamenta-se na aplicação adaptada do TDC. A principal diferença do presente trabalho está na integração do teste com o rastreamento ocular em tempo real, permitindo a coleta de dados visuais complementares durante a execução das tarefas.

O experimento é estruturado como um jogo digital de temática espacial, composto por
três fases com níveis crescentes de dificuldade. A mecânica de jogo foi desenhada para simular os
princípios do TDC, promovendo a exposição contínua a estímulos visuais por períodos prolongados e exigindo respostas rápidas e consistentes por parte do participante. Ao longo de cada fase, o sistema registra métricas relacionadas à atenção, como erros de omissão (quando o participante não responde a um estímulo-alvo), erros de comissão (quando não mantém foco por tempo suficiente no alvo), tempo de reação e variabilidade temporal das respostas.

Durante toda a experiência, o rastreamento ocular é realizado em segundo plano, utilizando
bibliotecas como o \textit{WebGazer} para capturar os pontos de fixação visual do usuário por meio da \textit{webcam}. Esses dados permitem identificar padrões de atenção ou desatenção de acordo com nossa base de dados em cada etapa da atividade. Todas as fases contam com música de fundo, cuja intensidade e ritmo são ajustados conforme o nível de dificuldade, de forma a potencializar a sobrecarga sensorial e dificultar a concentração.

A pesquisa de dados estatísticos foi conduzida por meio de um formulário composto por cinco questões, o qual foi respondido por cinco profissionais da área da Psicologia. No que se refere aos estímulos visuais empregados para a detecção de indícios de desatenção em crianças, os participantes destacaram que fatores como foco, tempo, som, cores e movimento constituem parâmetros relevantes para essa finalidade. Em virtude dessas considerações, tais elementos foram incorporados em todas as etapas do jogo desenvolvido. Ademais, todos os profissionais consultados reconheceram o rastreamento ocular como uma ferramenta útil, indicando que sua aplicação no contexto da abordagem proposta apresenta potencial para contribuir de forma eficaz na identificação de comportamentos relacionados à desatenção. Por fim, foram registradas sugestões de aprimoramento metodológico, como redirecionar o foco do estudo para uma aplicação de caráter clínico, inserir uma fase com estímulos \textit{distractors}, seguida de questionamentos específicos sobre o conteúdo \textit{distractor}, bem como assegurar que os estímulos sejam facilmente reconhecíveis pelas crianças participantes.

\begin{figure}[H]
    \centering
    \caption{Primeira fase}%
    \label{fig:primeira-fase}
    \includegraphics[width=\textwidth]{primeira-fase.png}%
    \SourceOrNote{Autoria Própria (2024)}
\end{figure}

Na primeira fase, o participante deve fixar o olhar por cinco segundos em cinco alvos estáticos,
representados por estrelas, enquanto elementos animados surgem ao redor. Após os 5 segundos,
cada estrela desaparece da tela. A música de fundo nesta etapa apresenta um ritmo moderado. O
objetivo é avaliar a capacidade de manter a atenção em um ponto fixo durante um tempo determinado, ignorando estímulos visuais e auditivos periféricos.

\begin{figure}[H]
    \centering
    \caption{Segunda fase}%
    \label{fig:segunda-fase}
    \SourceOrNote{Autoria Própria (2024)}
\end{figure}

Na segunda fase, são apresentadas cinco estrelas estáticas, que brilham individualmente em sequência, enquanto três planetas transitam pela tela, atuando como estímulos secundários que o participante deve acompanhar visualmente durante o movimento. Ao término da fase, o usuário responde a um formulário composto por perguntas relacionadas aos planetas observados, utilizando botões \textit{IoT}, cada um correspondente a um planeta específico. Essa etapa tem como objetivo mensurar a rastreabilidade ocular, exigindo que o participante direcione o olhar aos planetas em movimento contínuo pela tela. A trilha sonora torna-se mais intensa e acelerada nesta fase, com o propósito de aumentar o nível de exigência atencional e avaliar a capacidade de concentração diante de múltiplos estímulos visuais e auditivos.

\begin{figure}[H]
    \centering
    \caption{Terceira fase}%
    \label{fig:terceira-fase}
    \SourceOrNote{Autoria Própria (2024)}
\end{figure}

Na terceira fase, a demanda cognitiva é intensificada pela necessidade de alternância rápida do foco visual entre diferentes regiões da tela, caracterizadas por menor previsibilidade espacial. Nessa etapa, uma estrela surge de forma estática, exigindo resposta ocular imediata do participante. Simultaneamente, um segundo estímulo estático é apresentado, alternando entre os estados ligado e desligado em intervalos regulares. Quando esse estímulo é ativado (ascende), o participante deve manter o olhar fixo sobre ele até que se apague, o que permite avaliar a atenção sustentada e o controle do direcionamento ocular. A trilha sonora atinge seu nível máximo de intensidade e agitação, contribuindo para aumentar a complexidade da tarefa. O desempenho do participante nesta fase é utilizado como indicador da agilidade atencional e da capacidade de redirecionamento e manutenção do foco visual diante de estímulos dinâmicos.

Ao término das três fases, o sistema apresenta ao participante um resumo dos resultados com base em métricas como número de acertos, erros por omissão, erros por comissão e tempo médio de reação. Para isso, são utilizados os três registros mais recentes de rastreamento ocular, que correspondem às três fases concluídas pelo jogador. Com base nesses dados, o sistema gera um feedback textual interpretativo, apresentando mensagens como: “Sua atenção está conforme o esperado”, “Sua atenção está acima do esperado” ou “Sua atenção está abaixo do esperado.”, de acordo com o desempenho observado.

A plataforma é desenvolvida com tecnologias \textit{web}, permitindo acesso remoto e execução
direta em \textit{browsers} modernos. O teste é realizado de forma autônoma pelo usuário, em ambiente silencioso e seguindo instruções fornecidas pela própria plataforma.

Para melhor compreensão da sequência lógica do teste, a Figura \ref{fig:fluxograma-jogo} ilustra o fluxo de execução da plataforma desde o acesso inicial até a geração do feedback final. O \textit{flowchart} apresenta as etapas do experimento, desde o momento em que o usuário acessa o sistema e recebe instruções sobre o funcionamento do jogo, até a execução da primeira fase, baseada no \textit{Continuous Performance Test}. Nessa fase, são introduzidos estímulos visuais \textit{distractors}, e a plataforma registra o foco ocular do usuário utilizando visão computacional. A partir disso, os dados coletados — como desvios de atenção, acertos, erros e omissões — são analisados e comparados com uma base de dados para que o sistema gere um feedback interpretativo personalizado ao participante.

\begin{figure}[H]
    \centering
    \caption{Fluxograma do jogo}%
    \label{fig:fluxograma-jogo}
    \SourceOrNote{Autoria Própria (2024)}
\end{figure}

Optou-se pela utilização do banco de dados não relacional \textit{MongoDB}, o qual armazena informações em documentos no formato \textit{JSON}, possibilitando a criação de estruturas dinâmicas e aninhadas, adequadas ao armazenamento dos dados provenientes dos testes de rastreamento ocular. Sua flexibilidade e escalabilidade o tornam mais apropriado que bancos relacionais para o tratamento de grandes volumes de dados sensoriais. O gerenciamento do banco foi realizado por meio do \textit{MongoDB Compass}, ferramenta que facilita a execução de consultas, validação de esquemas e análise de desempenho.


% \section*{RESULTADOS PRELIMINARES}\label{sect:resultados}%

%\input{Topicos/resultados}%

%\section*{CONCLUSÃO}\label{sect:conclusao}%

%\input{Topicos/conclusao} %

\printbibliography

%% Elementos pós-textuais (opcionais): Apêndice e Anexo
%Caso for utilizar, basta retirar o símbolo de % na frente do comando
%\input{./Extras/post-textual}

%% Fim do documento
\end{document}