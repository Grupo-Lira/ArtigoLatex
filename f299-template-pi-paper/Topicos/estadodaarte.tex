O diagnóstico do Transtorno de Déficit de Atenção (TDA) em adultos continua sendo
um desafio clínico significativo. Tradicionalmente, o processo diagnóstico baseia-se em
entrevistas e testes clínicos, autorrelato e relatos de informantes, instrumentos que, embora
úteis, podem ser afetados por viés retrospectivo, subjetividade do paciente, e simulação dos
testes, resultando em casos de falsos positivos ou negativos. Dessa forma, o interesse por
abordagens objetivas e tecnologicamente assistidas, que combinem dados neuropsicológicos
e comportamentais com técnicas de análises automatizadas, aumentou. Estudos recentes
têm investigado o uso de jogos digitais sérios como ferramentas de avaliação e treinamento
cognitivo, com foco na atenção contínua. Nascimento e Menezes (2020) exploraram a relação
entre a prática regular de videogames e o desempenho em atenção sustentada, avaliado
pelo Conners’ Continuous Performance Test II (CPT II), uma versão amplamente utilizada do
Teste de Desempenho Contínuo (TDC). Embora não tenham encontrado diferenças de
performance entre jogadores de videogames de ação, não ação e não jogadores, os autores
identificaram o sexo como uma variável relevante, pois notara diferença entre tempo de
reação e número de erros. O estudo destaca a complexidade das interações entre fatores
individuais e experiências digitais, sugerindo que a aplicação de jogos, mesmo quando
classificados como serious games, precisam de cuidados na metodologia e no controle de
variáveis, com o fim de evitar interferências no desempenho atencional.

Nesse contexto, Elbaum et al. (2020) exploraram o potencial diagnóstico da integração
entre o MOXO-dCPT (um teste de desempenho contínuo com fases estruturadas de distração
auditiva e visual) e dados de rastreamento ocular (eye tracking). O estudo contou com uma
amostra de 85 adultos (43 com diagnóstico formal de TDAH e 42 controles saudáveis) e
analisou o padrão de atenção visual ao longo de diferentes partes do teste. Os resultados
demonstraram que indivíduos com TDAH apresentaram maior tempo de fixação em áreas
irrelevantes da tela, particularmente em condições com distrações visuais, o que os autores
interpretaram como uma medida direta de distratibilidade atencional objetiva. Essa métrica
comportamental demonstrou maior poder discriminativo em comparação às melhores
pontuações tradicionais do MOXO. Além disso, os autores propuseram que as partes do teste
com distrações visuais poderiam ser utilizadas isoladamente, reduzindo o tempo do teste e
mantendo a precisão.

Avançando nesse campo, Wiebe et al. (2024) criaram uma solução diagnóstica que
envolve um ambiente de realidade virtual (VR), onde os participantes realizavam um teste de
desempenho imersivo sob a ocorrência de distrações simuladas em um ambiente 3D.
Durante a tarefa, foram coletados dados simultâneos de eye tracking, movimentos da
cabeça, eletroencefalograma (EEG) e desempenho atencional. O modelo de IA foi treinado
em um conjunto de 50 participantes e testado de forma independente em outro conjunto de
36 indivíduos. O modelo final, com apenas 11 variáveis selecionadas, alcançou 81\% de
acurácia, 78\% de sensibilidade e 83\% de especificidade no conjunto de teste.

Os estudos indicam que a utilização de tecnologias de rastreamento ocular, tarefas
cognitivas com análises embasadas dos dados representam um avanço significativo em
relação aos métodos tradicionais de diagnóstico. O uso de serious games para coleta de
dados de desempenho também é eficaz, como mostra o trabalho de Nascimento e Menezes.
O trabalho de Elbaum et al. oferece um modelo aplicável e eficiente ao integrar eye tracking
em um teste comercial já existente, o estudo de Wiebe et al. diferencia a proposta ao
incorporar realidade virtual, aprendizado de máquina e validação externa em amostras
independentes. Juntos, os estudos reforçam a ideia de que sistemas digitais automatizados
podem melhorar a precisão diagnóstica do TDAH em adultos.

