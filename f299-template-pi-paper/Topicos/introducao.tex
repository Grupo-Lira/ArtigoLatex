A Organização das Nações Unidas (ONU) é uma instituição que visa estabelecer a paz,
segurança e desenvolvimento global. A ONU conta com 193 países membros que formam a
Assembleia Geral, responsável por desenvolver as políticas da organização. Em 2015, como
parte da Agenda 2030 para o Desenvolvimento Sustentável, foram criados 17 objetivos que
abrangem desde a melhoria da indústria até o aprimoramento da saúde da população. Esse
conjunto de metas constitui um plano de ação ambicioso para as pessoas, o planeta e a
prosperidade. Este trabalho visa contribuir para o terceiro objetivo, que busca garantir o
acesso à saúde e promover o bem estar, ao desenvolver uma plataforma digital gamificada
com o objetivo de auxiliar nos possíveis indícios do Transtorno de Déficit de Atenção (TDA)
em adultos.

O TDA é uma condição neurobiológica de causas genéticas, que afeta milhões de pessoas em todo o mundo. É caracterizada por sintomas de
desatenção, impulsividade, e, em alguns casos, hiperatividade, afetando significativamente o
desempenho acadêmico, profissional e social dos afetados. Embora o diagnóstico clínico do
TDA seja baseado tradicionalmente em entrevistas e questionários subjetivos, avanços
tecnológicos têm permitido o uso de ferramentas mais robustas e quantitativas para apoiar
esse processo. \textcite{BVS2014}
Entre essas ferramentas, destaca-se o \textit{eye tracking} (do português, rastramento ocular), esta é uma técnica
que consiste em usar o posicionamento dos olhos de uma pessoa para obter informações
sobre onde ela está olhando. Isso pode ser feito usando luzes infravermelhas, que calculam
exatamente onde a pessoa está olhando com base nas reflexões da luz na retina, ou por
meio de câmeras que monitoram visualmente a posição dos olhos e identificam sua direção.

Em ambientes controlados, o teste com \textit{eye tracking} envolve a realização de tarefas
padronizadas, nas quais o comportamento visual do participante é monitorado sem
interferência direta de um moderador. Essa abordagem objetiva permite uma coleta mais
confiável de dados, reduzindo o viés associado ao autorrelato e aumentando a credibilidade
da análise. Além disso, a comparação dos dados obtidos com padrões normativos permite
identificar desvios significativos no desempenho visual atencional, muitas vezes
imperceptíveis a métodos tradicionais.

A Inteligência Artificial (IA), é um campo da computação dedicado à criação de sistemas capazes de replicar a habilidade humana de executar tarefas que requerem percepção, raciocinio, aprendizado e tomada de decisão. A IA pode ser dividida em IA Forte e IA Fraca, \textcite{sep-chinese-room} define a IA Forte como programas que conseguem, além de entender a linguagem natural, também conseguem ter todas as características de raciocinio de um humano. A IA Fraca seria capaz de somente replicar tarefas específicas, e não possuir capacidade de raciocínio. A capacidade da IA de realizar tarefas é obtida através de algoritmos de aprendizagem profunda, processamento de linguagem natural e análise de dados.

O conceito de Internet das Coisas (do inglês, Internet of Things, ou IoT) vem da conexão de dispostivos fisícos (como sensores, dispositivos inteligêntes, eletrodomésticos, etc. ) à internet por meio de protocolos de comunicação, o que permite que os dados coletados sejam amplamente distribuídos e compartilhados de forma autônoma. A IoT integra computação em nuvem, comunicação de máquina e análise de dados para criar sistemas integrados e inteligentes. https://iot.ieee.org/
 
Quando aplicada em um teste diagnóstico, a técnica permite acompanhar com
precisão os movimentos dos olhos de um indivíduo durante a realização de tarefas
específicas, fornecendo dados objetivos e detalhados sobre onde, por quanto tempo e em
que sequência uma pessoa fixa seu olhar em determinados estímulos visuais. Estudos
demonstram que pessoas com TDA apresentam menor tempo de fixação e padrão visual
mais disperso ao realizarem testes de desempenho, sugerindo dificuldade em manter a
atenção sustentada e filtragem de estímulos irrelevantes \textcite{Lim2024}.

O Teste de Desempenho Contínuo (TDC) é uma medida padronizada amplamente
utilizada na neuropsicologia para avaliar métricas de atenção sustentada, impulsividade,
tempo de resposta e variabilidade dos tempos de reação. Trata-se de uma tarefa
computadorizada na qual o usuário responde e reage a estímulos apresentados
sequencialmente, permitindo medidas de desempenho atencional ao longo do tempo. Os
\textit{serious games}, ou jogos sérios, são aplicações digitais desenvolvidas com finalidades que
vão além do entretenimento, como educação, treinamento ou reabilitação cognitiva. No
contexto ne análises neuropsicológicas, eles vêm sendo utilizados como ferramentas
complementares aos testes tradicionais, oferecendo ambientes imersivos e interativos que
auxiliam no engajamento do usuário e a mensuração de habilidades cognitivas.

Com isso, a aplicação do \textit{eye tracking} em algo como os jogos sérios, podem servir não
apenas como ferramentas para complementar o diagnóstico clínico, mas também como um
potencial instrumento de triagem inicial. A análise dos dados com IA pode indicar o grau de
desatenção apresentado por um indivíduo em diferentes contextos e contribuir para decisões
clínicas mais embasadas. Assim, propomos por meio deste estudo a criação de um software
gamificado, que usa as informações obtidas pelo rastreamento ocular de possíveis afetados
pelo TDA e as processa, usando as métricas do TDC para gerar um resultado médio do
desempenho geral, fornecendo uma possível indicação para o transtorno.