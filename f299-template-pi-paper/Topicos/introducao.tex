A Organização das Nações Unidas (ONU) estabelece, por meio da Agenda 2030, metas para promover o desenvolvimento sustentável e o bem-estar global. Entre os 17 Objetivos de Desenvolvimento Sustentável (ODS), o terceiro busca garantir saúde de qualidade e promover o bem-estar para todos. Nesse contexto, este trabalho visa contribuir para essa meta ao propor uma solução tecnológica voltada à identificação de indícios do Transtorno de Déficit de Atenção (TDA) em adultos.\textcite{unitednations2015agenda2030}

O TDA é uma condição neurobiológica de causas genéticas, que afeta milhões de pessoas em todo o mundo. É caracterizada por sintomas de
desatenção, impulsividade, e, em alguns casos, hiperatividade, afetando significativamente o
desempenho acadêmico, profissional e social dos afetados. Embora o diagnóstico clínico do
TDA seja baseado tradicionalmente em entrevistas e questionários subjetivos, avanços
tecnológicos têm permitido o uso de ferramentas mais robustas e quantitativas para apoiar
esse processo. \textcite{BVS2014}
Entre essas ferramentas, destaca-se o \textit{eye tracking} (do português, rastramento ocular), esta é uma técnica
que consiste em usar o posicionamento dos olhos de uma pessoa para obter informações
sobre onde ela está olhando. Isso pode ser feito usando luzes infravermelhas, que calculam
exatamente onde a pessoa está olhando com base nas reflexões da luz na retina, ou por
meio de câmeras que monitoram visualmente a posição dos olhos e identificam sua direção.

Em ambientes controlados, o teste com \textit{eye tracking} envolve a realização de tarefas
padronizadas, nas quais o comportamento visual do participante é monitorado sem
interferência direta de um moderador. Essa abordagem objetiva permite uma coleta mais
confiável de dados, reduzindo o viés associado ao autorrelato e aumentando a credibilidade
da análise. Além disso, a comparação dos dados obtidos com padrões normativos permite
identificar desvios significativos no desempenho visual atencional, muitas vezes
imperceptíveis a métodos tradicionais.

Quando aplicada em um teste diagnóstico, a técnica permite acompanhar com
precisão os movimentos dos olhos de um indivíduo durante a realização de tarefas
específicas, fornecendo dados objetivos e detalhados sobre onde, por quanto tempo e em
que sequência uma pessoa fixa seu olhar em determinados estímulos visuais. Estudos
demonstram que pessoas com TDA apresentam menor tempo de fixação e padrão visual
mais disperso ao realizarem testes de desempenho, sugerindo dificuldade em manter a
atenção sustentada e filtragem de estímulos irrelevantes \textcite{Lim2024}.

A Inteligência Artificial (IA) surge como um campo essencial nesse contexto, voltado à criação de sistemas capazes de simular aspectos da cognição humana, como percepção, raciocínio, aprendizado e tomada de decisão. Segundo \textcite{sep-chinese-room}, a IA pode ser dividida em dois tipos: IA Forte, capaz de compreender e raciocinar de forma semelhante aos humanos; e IA Fraca, restrita à execução de tarefas específicas sem capacidade de raciocínio autônomo. O avanço dessas tecnologias, apoiado por algoritmos de aprendizagem profunda, processamento de linguagem natural e análise de dados, tem permitido desenvolver sistemas cada vez mais precisos e adaptativos.

Associada à IA, a Internet das Coisas (Internet of Things, IoT) amplia o potencial de integração tecnológica ao conectar dispositivos físicos, como sensores e equipamentos inteligentes, à internet. Essa interconexão possibilita a coleta, o compartilhamento e a análise de dados em tempo real, criando ecossistemas inteligentes baseados em computação em nuvem e comunicação entre máquinas. \textcite{IEEE}
 
O Teste de Desempenho Contínuo (TDC) é uma medida padronizada amplamente
utilizada na neuropsicologia para avaliar métricas de atenção sustentada, impulsividade,
tempo de resposta e variabilidade dos tempos de reação. Trata-se de uma tarefa
computadorizada na qual o usuário responde e reage a estímulos apresentados
sequencialmente, permitindo medidas de desempenho atencional ao longo do tempo. 
A atenção é um processo cognitivo complexo que envolve diferentes componentes inter-relacionados, essenciais para o desempenho em tarefas como o teste de desempenho contínuo. De forma geral, pode ser classificada em quatro tipos principais. A atenção sustentada refere-se à capacidade de manter o foco em um estímulo ou tarefa por um período prolongado, sendo fundamental para evitar erros de omissão no TDC. A atenção seletiva envolve a habilidade de concentrar-se em informações relevantes enquanto se ignora distrações, o que influencia diretamente a precisão das respostas. A atenção alternada diz respeito à flexibilidade cognitiva para mudar o foco entre diferentes estímulos ou tarefas, demonstrando controle executivo. Por fim, a atenção dividida representa a capacidade de processar simultaneamente múltiplas fontes de informação, exigindo coordenação eficiente de recursos cognitivos. A compreensão dessas modalidades permite interpretar de forma mais abrangente as métricas obtidas no TDC.\textcite{CULLUM1998303}

Os \textit{serious games} (jogos sérios) também vêm se destacando como ferramentas complementares em contextos de avaliação e reabilitação cognitiva. Esses jogos digitais, desenvolvidos com finalidades que vão além do entretenimento, oferecem ambientes interativos e imersivos que aumentam o engajamento do usuário e permitem mensurar habilidades cognitivas de forma dinâmica.

Desta forma,a integração entre eye tracking, IA e jogos sérios representa uma abordagem promissora para a triagem e o apoio diagnóstico do TDA.
A análise dos dados com IA pode indicar o grau de
desatenção apresentado por um indivíduo em diferentes contextos e contribuir para decisões
clínicas mais embasadas. Assim, este estudo propõe o desenvolvimento de um software gamificado que utilize informações de rastreamento ocular processadas com base nas métricas do TDC, a fim de gerar um indicador do desempenho atencional geral e oferecer uma possível estimativa preliminar para o Transtorno de Déficit de Atenção.