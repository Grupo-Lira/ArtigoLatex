A \textit{landing page} do projeto foi desenvolvida em \textit{HTML}, responsável pela estrutura do conteúdo, \textit{CSS}, utilizado para o estilo visual, e \textit{JavaScript}, empregado na implementação da interatividade e do dinamismo da navegação. O sistema de rastreamento ocular foi implementado em JavaScript, utilizando a biblioteca \textit{WebGazer.js}, que contém um modelo capaz de se autocalibrar ao observar a interação dos visitantes com a página, treinando um mapeamento entre as características do olhar e as posições na tela \cite{papoutsaki2016webgazer}. O tratamento das coordenadas oculares recebidas do \textit{frontend} foi realizado em JavaScript, com o uso do \textit{Node.js} e do \textit{Socket.IO}, possibilitando a comunicação em tempo real com o frontend, uma vez que depende das coordenadas enviadas por ele. Essa parte do \textit{backend} é responsável por analisar as métricas TDC (acertos e erros), dados que serão utilizados para compor o feedback individual de cada usuário. Por fim, o jogo \textit{web} foi desenvolvido em \textit{TypeScript}, utilizando o \textit{framework} \textit{Next.js}, o que proporcionou um código mais robusto, organizado e uma experiência de uso moderna e fluida.

A metodologia fundamenta-se na aplicação adaptada do TDC. A principal diferença do presente trabalho está na integração do teste com o rastreamento ocular em tempo real, permitindo a coleta de dados visuais complementares durante a execução das tarefas.

O experimento é estruturado como um jogo digital de temática espacial, composto por
três fases com níveis crescentes de dificuldade. A mecânica de jogo foi desenhada para simular os
princípios do TDC, promovendo a exposição contínua a estímulos visuais por períodos prolongados e exigindo respostas rápidas e consistentes por parte do participante. Ao longo de cada fase, o sistema registra métricas relacionadas à atenção, como erros de omissão (quando o participante não responde a um estímulo-alvo), erros de comissão (quando não mantém foco por tempo suficiente no alvo), tempo de reação e variabilidade temporal das respostas. COLOCAR REF

Durante toda a experiência, o rastreamento ocular é realizado em segundo plano, utilizando a biblioteca \textit{WebGazer} para capturar os pontos de fixação visual do usuário por meio da \textit{webcam}. Esses dados permitem identificar padrões de atenção ou desatenção de acordo com nossa base de dados em cada etapa da atividade. Todas as fases do experimento são configuradas para potencializar a sobrecarga sensorial e dificultar a concentração do participante. Para tal, utilizam-se música de fundo — cuja intensidade e ritmo são ajustados conforme o nível de dificuldade — e a imposição de um tempo máximo definido, elementos que, em conjunto, intensificam o estresse cognitivo e a exigência da tarefa.

Após o realizar \textit{login}, o participante é direcionado para a página de instruções, onde são apresentadas a sequência de como calibrar o olhar para poder prosseguir para o jogo. Antes de cada fase, existem as instruções da mesma. Em seguida, o usuário inicia a primeira etapa do teste. 

\begin{figure}[H]
    \centering
    \caption{Primeira fase}%
    \label{fig:primeira-fase}
    \includegraphics[width=\textwidth]{primeira-fase.png}%
    \SourceOrNote{Autoria Própria (2025)}
\end{figure}

Na primeira fase, o participante deve fixar o olhar por cinco segundos em cinco alvos estáticos,
representados por estrelas, enquanto elementos animados surgem ao redor. Após os 5 segundos,
cada estrela desaparece da tela. A música de fundo nesta etapa apresenta um ritmo moderado. O
objetivo é avaliar a capacidade de manter a atenção em um ponto fixo durante um tempo determinado, ignorando estímulos visuais e auditivos periféricos.

\begin{figure}[H]
    \centering
    \caption{Segunda fase}%
    \label{fig:segunda-fase}
    \includegraphics[width=\textwidth]{segunda-fase.png}%
    \SourceOrNote{Autoria Própria (2025)}
\end{figure}

Na segunda fase do experimento, o participante é instruído a manter o foco em cinco estrelas estáticas que brilham individualmente em sequência (o estímulo primário). Simultaneamente, três planetas em movimento transitam pela tela, atuando como distratores (estímulos secundários) cuja presença não é mencionada nas instruções e que, idealmente, devem ser ignorados. Ao término da exibição, o participante é submetido a um teste de reconhecimento: sete opções de planetas são apresentadas, sendo que apenas três transitaram na tela. O participante deve indicar, por meio de botões IoT, quais planetas foram reconhecidos durante o trânsito.
O objetivo central dessa etapa é mensurar a capacidade de inibição de respostas a estímulos irrelevantes. A expectativa é que indivíduos com padrão de atenção neurotípica mantenham o foco na tarefa principal das estrelas e, portanto, errem a identificação dos planetas (ou seja, não os reconheçam). Por outro lado, o acerto na identificação do planeta (o reconhecimento) é interpretado como um erro de inibição atencional — uma dificuldade de suprimir o estímulo secundário e irrelevante —, característica de perfis com potencial TDAH.
A trilha sonora, que se torna gradualmente mais intensa e acelerada, amplia a carga cognitiva e permite observar como múltiplos estímulos simultâneos afetam a manutenção da atenção.

Para a construção do dispositivo de controle da \textit{IoT}, foram empregados os seguintes componentes eletrônicos: uma placa de desenvolvimento Arduino, três botões do tipo push-button, três resistores de 10 k$\Omega$, uma protoboard e cabos de conexão. Os botões foram conectados às portas digitais do Arduino, configuradas como entradas com resistores pull-down para garantir leituras estáveis. Dessa forma, ao término da segunda fase do jogo, o participante registra os planetas que conseguiu observar, pressionando o botão correspondente a ele. O Arduino atua como uma interface de comunicação de hardware, detectando o pressionamento físico dos botões IoT. Por meio da comunicação serial, o microcontrolador transmite o evento acionado ao sistema. O servidor é responsável por receber e interpretar este sinal do botão selecionado e, na sequência, acionar a função que contabiliza os acertos e os erros de inibição do participante para a fase.
Em futuras iterações do projeto, planeja-se o aperfeiçoamento estético e ergonômico do controle, por meio da substituição dos botões convencionais por peças personalizadas, modeladas em software CAD e fabricadas via impressão 3D, com design mais atrativo e ergonômico para crianças. Além disso, planeja-se construir uma case para o controle utilizando filamentos de garrafa PET, visando a sustentabilidade ambiental do projeto.

\begin{figure}[H]
    \centering
    \caption{Terceira fase}%
    \label{fig:terceira-fase}
    \includegraphics[width=\textwidth]{terceira-fase.png}%
    \SourceOrNote{Autoria Própria (2025)}
\end{figure}

Na terceira fase, a demanda cognitiva é intensificada pela necessidade de alternância rápida do foco visual entre diferentes regiões da tela, caracterizadas por menor previsibilidade espacial. Nessa etapa, uma estrela surge de forma estática, exigindo resposta ocular imediata do participante. Simultaneamente, um segundo estímulo estático é apresentado, alternando entre os estados ligado e desligado em intervalos regulares. Quando esse estímulo é ativado (ascende), o participante deve manter o olhar fixo sobre ele até que se apague, o que permite avaliar a atenção sustentada e o controle do direcionamento ocular. A trilha sonora atinge seu nível máximo de intensidade e agitação, contribuindo para aumentar a complexidade da tarefa. O desempenho do participante nesta fase é utilizado como indicador da agilidade atencional e da capacidade de redirecionamento e manutenção do foco visual diante de estímulos dinâmicos.

Ao final de cada fase, o servidor é responsável por enviar ao cliente (front-end) a quantidade de acertos, erros de omissão e erros de comissão atingidos pelo participante.

Ao término das três fases, o sistema gera um pré-diagnóstico com base nas métricas TDC. Para isso, são utilizados os registros de desempenho e os dados mais recentes de rastreamento ocular obtidos durante as fases. Este pré-diagnóstico é elaborado por um módulo de Inteligência Artificial (IA), que interpreta as métricas e o desempenho do participante. O feedback resultante é apresentado em duas formas: (1) uma mensagem textual interpretativa, como: “Sua atenção está conforme o esperado”, “Sua atenção está acima do esperado” ou “Sua atenção está abaixo do esperado”, de acordo com o desempenho observado; e (2) a exibição de uma porcentagem que representa a precisão do olhar capturada pelo sistema de rastreamento ocular.

Com o objetivo de aprofundar a compreensão do tema e promover um alinhamento mais preciso entre os aspectos técnicos e psicológicos do projeto, foi conduzida uma pesquisa de campo com quatro profissionais da área da Psicologia. As entrevistas buscaram avaliar a viabilidade técnica da proposta, identificar os sintomas mais recorrentes do TDA e compreender os tipos de estímulos que mais influenciam os processos de distração em indivíduos com esse transtorno.

A análise qualitativa dos dados revelou que estímulos visuais e auditivos, especialmente movimento, cor e som, exercem papel determinante na indução de distrações, devendo, portanto, ser considerados elementos centrais no planejamento das fases experimentais. Essa lógica foi implementada em todas as fases do projeto.
Além disso, os profissionais destacaram a importância de incluir uma etapa em que os distratores estejam presentes de forma indireta, sem que o participante seja instruído a focar neles.

A plataforma é desenvolvida com tecnologias \textit{web}, permitindo acesso remoto e execução
direta em \textit{browsers} modernos. O teste é realizado de forma autônoma pelo usuário, em ambiente silencioso e seguindo instruções fornecidas pela própria plataforma.

No projeto atual, a IA será utilizada para análisar os dados coletados durante o jogo, referentes aos erros e acertos, com uma base de dados previamente formada por indivíduos diagnosticados com TDA e por outros sem o transtorno. Entretanto, nesta fase inicial de desenvolvimento, é necessário testar a viabilidade do sistema. Inicialmente, dez crianças com TDA serão convidadas a participar do experimento, com o objetivo de coletar dados iniciais que servirão como base para o treinamento supervisionado do modelo de IA. Em seguida, três crianças com TDA e três sem o transtorno (diferentes das 10 iniciais, ou seja, ainda não analisadas) 
serão convidadas para uma nova etapa experimental, destinada a validar a eficácia do sistema na identificação de indícios de desatenção. O objetivo é garantir que a acurácia do sistema seja satisfatória antes de ampliar a base de dados e aperfeiçoar o modelo de IA. Após a conclusão da fase de validação, o sistema estará apto a ser utilizado por um público mais amplo, contribuindo para a identificação precoce do TDA em crianças e reforçando seu potencial como ferramenta de apoio ao diagnóstico. 

No que se refere à retroalimentação da IA, o sistema irá conter um modo de treinamento, que pode ser ativado ou desativado exclusivamente pelo administrador. A mecânica dessa funcionalidade é empregada sempre que houver necessidade de alimentar a base de dados com novos registros. Essa etapa só pode ser executada na presença de ao menos um administrador, a fim de garantir a integridade e a qualidade dos dados inseridos. Caso contrário, a criança participa normalmente do jogo apenas para avaliar seu nível de atenção. Quando o modo de treinamento está habilitado, o sistema armazena as métricas TDC coletadas durante as partidas em uma base de dados, permitindo que o módulo de treinamento da IA realize a análise comparativa entre os dados do indivíduo e a base existente. Esse processo tem como objetivo retroalimentar o modelo e aperfeiçoar continuamente o desempenho da IA. 

Para uma melhor compreensão do funcionamento do sistema, a Figura \ref{fig:fluxograma} apresenta o fluxograma do processo, que descreve a sequência lógica das operações realizadas pelo sistema até o término das três partidas. Na etapa 1, ocorre a calibração dos olhos do jogador, processo em que o sistema identifica e ajusta os pontos de fixação ocular do usuário antes do início do jogo, garantindo a precisão do rastreamento. Em seguida, é executada a coleta das métricas TDC, que ocorre automaticamente durante as três fases do jogo. Nessa etapa, o sistema processa e registra parâmetros como número de acertos, erros de omissão, erros de comissão, tempo de reação e variabilidade temporal das respostas. Na etapa 2, há uma tomada de decisão para verificar se o participante irá jogar no modo de treinamento. Caso o modo de treinamento esteja habilitado (etapa 3), o sistema armazena as métricas TDC em uma base de dados, permitindo que o módulo de treinamento da IA realize a análise comparativa entre os dados do indivíduo e a base existente. Esse processo visa aprimorar o modelo e gerar um pré-diagnóstico personalizado. Já na etapa 4, quando o modo de treinamento não está ativado, o sistema realiza diretamente a geração e exibição do pré-diagnóstico, utilizando as métricas coletadas durante a execução do jogo.

\begin{figure}[H]
    \centering
    \caption{Fluxograma do sistema}%
    \label{fig:fluxograma}
    \includegraphics[width=\textwidth]{fluxograma.png}%
    \SourceOrNote{Autoria Própria (2025)}
\end{figure}

Optou-se pela utilização do banco de dados não relacional \textit{MongoDB}, o qual armazena informações em documentos no formato \textit{JSON}, possibilitando a criação de estruturas dinâmicas e aninhadas, adequadas ao armazenamento dos dados provenientes dos testes de rastreamento ocular. Sua flexibilidade e escalabilidade o tornam mais apropriado que bancos relacionais para o tratamento de grandes volumes de dados sensoriais. O gerenciamento do banco foi realizado por meio do \textit{MongoDB Compass}, ferramenta que facilita a execução de consultas, validação de esquemas e análise de desempenho.
