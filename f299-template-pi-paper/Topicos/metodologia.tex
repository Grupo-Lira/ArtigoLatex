Este projeto propõe o desenvolvimento de uma plataforma digital gamificada com o objetivo
de auxiliar nos possíveis indícios do Transtorno de Déficit de Atenção (TDA) em adultos. A
metodologia fundamenta-se na aplicação adaptada do Teste de Desempenho Contínuo
(TDC), tradicionalmente utilizado em avaliações neuropsicológicas para medir aspectos da
atenção sustentada, impulsividade, tempo de resposta e variabilidade das reações. \textcite{CPT2025}. A
principal diferença do presente trabalho está na integração do TDC com o rastreamento
ocular em tempo real, o que permite a coleta de dados visuais complementares durante a
execução das tarefas.

O experimento foi estruturado como um jogo digital de temática espacial, composto por três
fases com níveis crescentes de dificuldade. A mecânica de jogo foi desenhada para simular
os princípios do TDC, promovendo a exposição contínua a estímulos visuais por períodos
prolongados e exigindo respostas rápidas e consistentes por parte do participante. Ao longo
de cada fase, o sistema registra métricas relacionadas à atenção, como erros de omissão
(quando o participante não responde a um estímulo-alvo), erros de comissão (quando
responde a estímulos irrelevantes), tempo de reação e variabilidade temporal das respostas.

Durante toda a experiência, o rastreamento ocular é realizado em segundo plano, utilizando
bibliotecas como o MediaPipe para capturar os pontos de fixação visual do usuário por meio
da webcam. Esses dados permitem identificar padrões de atenção ou desatenção de acordo
com nossa base de dados em cada etapa da atividade. Todas as fases contam com música
de fundo, cuja intensidade e ritmo são ajustados conforme o nível de dificuldade, de forma a
potencializar a sobrecarga sensorial e dificultar a concentração.

Na primeira fase, o participante deve fixar o olhar por cinco segundos em cinco alvos
estáticos, representados por estrelas, enquanto elementos animados surgem ao redor. A
música de fundo nesta etapa apresenta um ritmo moderado. O objetivo é avaliar a
capacidade de manter a atenção em um ponto fixo durante um tempo determinado,
ignorando estímulos visuais e auditivos periféricos.

Na segunda fase, introduzem-se estímulos dinâmicos, visando mensurar a rastreabilidade
ocular. O participante deve acompanhar com o olhar cinco estrelas em movimento contínuo
pela tela. A música, nesta etapa, torna-se mais intensa, com ritmo acelerado, a fim de
aumentar o desafio atencional.

Na terceira fase, a demanda cognitiva é acentuada por meio da exigência de alternância
rápida do foco visual entre diferentes regiões da tela, com menor previsibilidade espacial.
Dez estrelas surgem, uma por vez, em locais variados e por tempo limitado, exigindo
resposta ocular imediata. A música de fundo atinge seu nível máximo de agitação,
intensificando a dificuldade da tarefa. O desempenho nesta fase é utilizado como indicador
da agilidade atencional e da capacidade de redirecionamento rápido do foco visual.
Ao término das três fases, o sistema apresenta ao participante um resumo dos resultados
com base em métricas como número de acertos e tempo médio de reação. Para isso, são
utilizados os três registros mais recentes de rastreamento ocular, que correspondem às três
fases concluídas pelo jogador. Com base nesses dados, o sistema gera um feedback textual
interpretativo, apresentando mensagens como: “Sua atenção está conforme o esperado”,
“Sua atenção está acima do esperado” ou “Sua atenção está abaixo do esperado.”, de acordo
com o desempenho observado.

A plataforma é desenvolvida com tecnologias web, permitindo acesso remoto e execução
direta em navegadores modernos. O teste é realizado de forma autônoma pelo usuário, em
ambiente silencioso e seguindo instruções fornecidas pela própria plataforma.

Para melhor compreensão da sequência lógica do teste, a Figura 1 ilustra o fluxo de
execução da plataforma desde o acesso inicial até a geração do feedback final. O fluxograma
apresenta as etapas do experimento, desde o momento em que o usuário acessa o sistema e
recebe instruções sobre o funcionamento do jogo, até a execução da primeira fase, baseada
no Teste de Desempenho Contínuo. Nessa fase, são introduzidos estímulos visuais
distratores, e a plataforma registra o foco ocular do usuário utilizando visão computacional. A
partir disso, os dados coletados — como desvios de atenção, acertos, erros e omissões —
são analisados e comparados com uma base de dados para que o sistema gere um feedback
interpretativo personalizado ao participante.

\begin{figure}[H]
    \centering
    \caption{Fluxo da Plataforma}%
    \label{fig:fluxograma-plataforma}
    \includegraphics[width=\textwidth]{fluxogramametodologia}
    \SourceOrNote{Autoria Própria (2024)}
\end{figure}

Para desenvolver a landing page da equipe do projeto, foi utilizada a linguagem de marcação
HyperText Markup Language (HTML), responsável por definir a estrutura e o conteúdo
principal da página. A estilização visual foi realizada com Cascading Style Sheets (CSS), o que
permitiu aplicar cores, espaçamentos, tipografias e outros elementos de design, assegurando
que a página estivesse visualmente alinhada com a identidade da equipe. Além disso,
utilizou-se JavaScript, que introduziu interatividade e dinamismo à página, proporcionando
animações e ações responsivas, aprimorando a experiência de navegação e tornando o
conteúdo mais atraente para o usuário.

No desenvolvimento do sistema de rastreamento ocular (Eye Tracking), optou-se pela
linguagem Python devido à sua compatibilidade com bibliotecas de visão computacional e
análise de dados, bem como à sua eficiência e facilidade de aprendizado. Python permite a
criação de programas com menos linhas de código, o que aumenta a produtividade dos
desenvolvedores e agiliza o desenvolvimento. Além disso, sua grande biblioteca padrão
contém diversos módulos reutilizáveis, que eliminam a necessidade de escrever código do
zero para muitas tarefas.

Com o intuito de capturar imagens em tempo real da câmera, utilizou-se a biblioteca OpenCv,
e o framework MediaPipe Face Mesh foi escolhido para a detecção facial. O Face Mesh do
MediaPipe identifica 468 landmarks (pontos de referência) na face, mapeando características
como olhos, boca, nariz e contorno facial. Esse mapeamento detalhado permite rastrear
micro movimentos da íris, possibilitando identificar a direção e o foco visual em tempo real.

Além dessas ferramentas, a biblioteca NumPy foi utilizada para estruturar os dados de
coordenadas em matrizes, e a biblioteca JSON auxiliou no armazenamento dos dados.
Python ainda oferece a vantagem de portabilidade, podendo ser executado em diversos
sistemas operacionais como Windows, macOS e Linux. Sua ampla comunidade global de
suporte facilita o aprendizado e proporciona soluções rápidas a problemas, contribuindo para
a manutenção e evolução contínua do sistema.

Além dos elementos já mencionados, para o desenvolvimento do jogo web, a equipe optou
por utilizar TypeScript no front-end, uma linguagem baseada em JavaScript que adiciona
tipagem estática ao código, tornando-o mais robusto e facilitando a identificação de erros
durante o desenvolvimento. A estrutura do front-end foi construída com o framework Next.js,
que permite a criação de aplicações web modernas com renderização híbrida (estática e
dinâmica), roteamento automático e excelente desempenho. O uso combinado de TypeScript
e Next.js proporcionou uma base sólida para o desenvolvimento da interface do jogo,
garantindo maior escalabilidade, organização do código e melhor experiência do usuário.

Para o desenvolvimento dos códigos mencionados, utilizou-se o Visual Studio Code (VS
Code), um Ambiente de Desenvolvimento Integrado (IDE) criado pela Microsoft em 2015. O
VS Code é um editor de código aberto amplamente reconhecido e utilizado na comunidade
de desenvolvimento por sua versatilidade e eficiência. Ele suporta diversas linguagens de
programação, oferecendo uma interface amigável e funcionalidades que potencializam o
processo de desenvolvimento. \textcite{HanashiroVSCode}

Em relação a diagramação, foi utilizada a Unified Modeling Language (UML), desenvolvida
por Grady Booch, James Rumbaugh e Ivar Jacobson, que serve para documentar projetos de
software. A UML pode ser usada para visualizar, especificar, construir e documentar os
artefatos de um sistema de software. No atual projeto, foram construídos dois diagramas
UML, sendo eles o Diagrama de Classes, o Diagrama de Objetos e o Diagrama de Caso de
Uso.

O diagrama de classes é uma representação visual da estrutura de um sistema, descrevendo
classes, seus atributos, métodos e os relacionamentos entre elas. Enquanto o Diagrama de
Objetos é uma representação visual que exibe instâncias específicas de classes em um
momento particular do sistema, mostrando objetos com seus valores atuais de atributos e as
relações entre eles. Diferente do diagrama de classes, que é mais abstrato e define a
estrutura geral, o diagrama de objetos detalha uma visão concreta do estado do sistema em
determinado instante. Por fim, o Diagrama de Caso de Uso serve para ilustrar os atores e
como eles interagem com o sistema, a fim de definir requisitos funcionais e o seu contexto.
Apesar de não ser tão detalhado como os demais, utiliza diversos símbolos para contribuir no
fluxo básico dos eventos.

Para a diagramação dos modelos UML no projeto, foi utilizada a ferramenta LucidChart. O
LucidChart é uma plataforma de criação de diagramas baseada na nuvem, amplamente
usada para projetar e documentar sistemas complexos de forma visual.

O Diagrama de Infraestrutura de Rede é uma representação visual da arquitetura de
comunicação e conectividade entre os componentes de hardware e software de um sistema.
Esse tipo de diagrama é essencial para compreender como os dados trafegam na rede,
identificar pontos críticos e garantir a segurança, a escalabilidade e a eficiência da
infraestrutura.

Para a elaboração e documentação da infraestrutura de rede no projeto, foi utilizada a
ferramenta draw.io. O draw.io é uma aplicação gratuita baseada na web, amplamente
utilizada para criar diagramas técnicos, incluindo fluxogramas, topologias de rede, diagramas
UML e arquiteturas de sistemas.

Em relação à elaboração do banco de dados, optou-se pelo uso de um banco de dados não
relacional, mais especificamente o MongoDB. O MongoDB adota o formato de documentos
no padrão JSON (JavaScript Object Notation), permitindo que os dados sejam armazenados
em estruturas aninhadas e dinâmicas, o que se mostrou especialmente vantajoso para a
manipulação de informações provenientes dos testes de rastreamento ocular.

Diferentemente dos bancos relacionais, que organizam os dados em tabelas com colunas
fixas, o MongoDB permite a criação de documentos com campos distintos dentro da mesma
coleção, o que torna o sistema mais escalável e flexível, especialmente para aplicações que
lidam com grandes volumes de dados sensoriais.

Para o desenvolvimento do banco de dados e gerenciamento das coleções, foi utilizada a
ferramenta MongoDB Compass, uma interface gráfica oficial do MongoDB que permite a
visualização, consulta e manipulação dos dados de forma intuitiva. A ferramenta facilita a
inspeção de documentos, a criação de filtros avançados utilizando a linguagem de consulta
do MongoDB (MQL - MongoDB Query Language), além de oferecer suporte para validação
de esquema e análise de desempenho.

Para desenvolver a logo e a identidade visual da equipe e do jogo, bem como a prototipação
das telas do software, foi utilizada a plataforma colaborativa de design Figma. Essa
ferramenta permite criar interfaces e protótipos de forma intuitiva, possibilitando a
construção de fluxos de navegação e layouts de maneira integrada. Além disso, o Figma
favorece a colaboração em tempo real entre designers e demais membros da equipe, o que
contribuiu para uma criação mais eficiente e alinhada com os objetivos do projeto.\textcite{VillainSilveiraFigma}

Além disso, utilizamos a ferramenta SEBRAE Canvas para estruturar o modelo de negócios do
jogo. Um modelo de negócios descreve como uma organização cria, entrega e captura valor,
abordando os principais componentes que influenciam o sucesso de uma empresa. Essa
ferramenta oferece um quadro segmentado em áreas como parcerias, proposta de valor,
estrutura de custos e fontes de receita, facilitando o planejamento de cada componente do
modelo de negócios e a definição dos elementos essenciais do projeto. Vale ressaltar que o
Canvas foi criado antes do desenvolvimento do jogo, pois ele proporciona uma visão clara e
abrangente do sistema.

A análise SWOT foi realizada com o objetivo de identificar de forma estratégica os pontos
fortes, pontos fracos, oportunidades e ameaças relacionadas ao desenvolvimento do
software. Essa técnica permitiu uma visão abrangente do contexto interno e externo do
projeto, servindo como base para a definição de estratégias mais eficazes e para a tomada
de decisões ao longo do processo de desenvolvimento.
Para a organização e gestão de tarefas durante o desenvolvimento do sistema, foram
adotadas metodologias ágeis e ferramentas que favorecem o trabalho colaborativo e a
entrega eficiente de valor.

Dentre as metodologias utilizadas, destaca-se o Scrum, um framework de gerenciamento ágil
que permite que equipes se auto-organizem e trabalhem em direção a um objetivo comum.
Com o Scrum, foram incorporados valores fundamentais como compromisso, coragem, foco,
abertura e respeito. As atividades foram estruturadas em sprints semanais, com a aplicação
dos principais artefatos da metodologia, como o backlog, a sprint backlog e as reuniões. \textcite{Scrum2022}

Outra metodologia adotada foi o Kanban, um sistema visual originado na Toyota, no Japão,
que visa otimizar o fluxo de trabalho. O Kanban organiza as tarefas em colunas como “A
Fazer”, “Em Progresso” e “Concluído”, oferecendo uma visão clara do andamento do projeto e
das prioridades. \textcite{TOTVSKanban}

Para apoiar a aplicação dessas metodologias, foi utilizada a ferramenta Trello, um sistema de
gerenciamento de projetos baseado em quadros visuais. Por meio do Trello, as tarefas foram
organizadas em cartões, que podiam ser movidos entre listas, proporcionando um
acompanhamento visual do progresso. Cada cartão permitia a inclusão de checklists, datas
de vencimento e responsáveis, oferecendo uma visão detalhada de cada etapa do projeto.
Assim, a combinação entre o Trello e a metodologia Kanban contribuiu significativamente
para a organização, eficiência e colaboração no desenvolvimento do sistema. \textcite{MagalhaesTrello} 
